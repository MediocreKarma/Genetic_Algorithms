\documentclass{article}

\usepackage{graphicx}
\usepackage{hyperref}
\usepackage{tikz}
\usepackage{pgfplots}
\usepackage{siunitx}
\usepackage{float}
\usepackage{amsfonts}
\newcommand{\euler}{e}

\author{Chirvasa Matei \& Rotariu George}
\title{Analysis of different heuristic methods of determining the global minimum of a function}

\begin{document}
\maketitle

\section{Abstract}
In this article we will assess the effectiveness of different probabilistic methods in determining the global minimum of a function. The studied methods were Hill Climbing, with different improvement variants, namely best, first and worst improvements, and Simulated Annealing using a Geometric Arithmetic cooling method. We will observe a higher degree of variance in the results generated by the Simulated Annealing method, but with much faster runtime and similar, if not better, results compared to the Hill Climbing methods. \\
Key words: Simulated Annealing, Hill Climbing, Global Optimization

\section{Introduction}
We will study two commonly used heuristic methods, with the first being \underline{\href{https://en.wikipedia.org/wiki/Hill_climbing}{Hill Climbing}}\cite{HC}, tested using three improvement methods. The best improvement method, also known as steepest ascent hill climbing, compares all neighbouring solutions to a problem and picks the one that makes most progress towards the optimization goal, whereas the first improvement method will choose the the first value that makes progress towards the optimization goal. The worst improvement method is counter-intuitive, as it will select the neighbouring value that makes least progress towards the optimization goal, but isn't worse than the previously selected value. The hill climbing algorithm will find local optima to a given function, and as such, the functions containing only the global optima will be solved with a probability $p=1$. \\
The second algorithm to analyse in this paper is the \underline{\href{https://en.wikipedia.org/wiki/Simulated_annealing}{Simulated annealing}}\cite{SA} algorithm, named after the heat treatment used on metals in order to make them more workable, and then slowly letting the metal cool at room temperature. The main difference between the two is that simulated annealing is capable of exploring suboptimal neighbours of the current solution in order to get past the local optima limitations that the hill climbing algorithm faced. The cooling method can greatly modify the efficacy of the algorithm, but will not affect the actual implementation. A random neighbouring value will be chosen, and if the value makes progress towards the optimization goal it is selected, otherwise, given an implementation defined probability that relies on the current temperature, the value may or may not be chosen, in order to find a better optima.

\subsection{Motivation}
The problem of finding the global optimum of a function appears in many other problems in our field, including AI and \underline{\href{https://en.wikipedia.org/wiki/Combinatorial_optimization}{NP-Optimization problems}}, and as such developing efficient and effective algorithms to solve these problems became a necessity in our field. It became readily apparent that the tried and tested deterministic methods of solving such problems were either too slow to serve any purpose, or the solutions rendered were too imprecise to have any merit. This lead to the development of heuristic methods in order to, at the very least, approximate a solution to the problem.

\section{Method}
\subsection{Functions}
The heuristic algorithms were applied on the following multivariate functions: \underline{\href{http://www.geatbx.com/docu/fcnindex-01.html\#P89_3085}{De Jong's Sphere function}}, \underline{\href{http://www.geatbx.com/docu/fcnindex-01.html\#P150_6749}{Schwefel's function}}, \underline{\href{http://www.geatbx.com/docu/fcnindex-01.html\#P140_6155}{Rastrigin's function}} and \\\underline{\href{http://www.geatbx.com/docu/fcnindex-01.html\#P204_10395}{Michalewicz's function}}\cite{Functions}.
\\ \\
De Jong's Sphere function:
$$ f(x) = \sum_{i = 1}^n x_i^2, 
x_i \in \left[ -5.12, 5.12 \right] $$

Schwefel's function:
$$ f(x) = \sum_{i = 1}^n -x_i \cdot sin\left(\sqrt{\left|x_i\right|}\right), 
x_i \in \left[ -500, 500 \right] $$

Rastrigin's Function:
$$ f(x) = A \cdot n + \sum_{i=1}^n \left[ x_i^2 - A \cdot cos(2 \pi x_i) \right],
A = 10, x_i \in \left[ -5.12, 5.12 \right]$$

Michalewicz's Function:
$$ f(x) = - \sum_{i=1}^n sin(x_i) \cdot sin^{2m}\left(\frac{i \cdot x_i^2}{\pi}\right),
m = 10, x_i \in \left[ 0, \pi \right]$$

In order to analyse the previously mentioned algorithms, we will consider the functions in their 5, 10 and 30-dimensional forms.

\subsection{Algorithms}
In order to obtain any meaningful information about our results, we have to establish certain parameters. We shall set the explorable values of a domain $\text{[infimum, supremum]}^n$, to be the set $S_n$, where:
$$ S_n = \left\{ x | x \in [\text{infimum}, \text{supremum}]^n, x = (\text{infimum})^n + a \cdot \varepsilon, a \in \mathbb{N}^n \right\}, \varepsilon = \num{e-5} $$
Simply put, the minimal step between two values of a domain must be at least $\varepsilon$ for at least one dimension of the value, and the value $ x_0 = (\text{infimum})^n $ must be included in the set of possible values. This limitation is imposed in order to avoid excessively long computation times, as this study is meant to show the power of the heuristic approaches, and finer detail may be used when the results convey meaningful information other than a proof of concept.
The implementation of both algorithms will interpret the domain values using their bit representation, and as such we have chosen the neighbours of any sequence of bits to be any other sequence of bits such that their \underline{\href{https://en.wikipedia.org/wiki/Hamming_distance}{Hamming distance}}\cite{HD} be 1.
$$ \text{Let } a \in \left\{0, 1\right\}^n,
\text{Neighbours} \colon \mathbb{B}^n \to 2^{\mathbb{B}^n},
\text{HD}:\mathbb{B}^n \times \mathbb{B}^n \to \mathbb{N},$$ $$\text{where HD is the Hamming Distance function.}$$
$$ \text{Neighbours}\left(x_0\right) = \left\{x | x \in \mathbb{B}^n, \text{HD}\left(x, x_0\right) = 1  \right\} $$
This representation leads to some consequences, particularly when dealing with neighbouring values. When tackling real numbers, the neighbouring values are similar, whereas bit strings at a Hamming Distance of 1 may have a difference of as much as $ \frac{1}{2}\cdot\left(\text{supremum} - \text{infimum}\right)$ in their real number representation. This will lead to different local optima compared to those we may expect when using real number representations, and as such the local optima in which the Hill Climbing algorithm may get `stuck` are different from the local optima of the function in most cases. \\
The hill climbing algorithm, irrespective of improvement method, will run in linear time per value, in order to determine it's neighbours. We can assume that the evaluation and bit conversion functions run in constant time, as the amount of bits needed to represent a value are known at compile-time. In theory, runtime complexity of the hill climbing algorithm rivals that of a deterministic, exhaustive search, but in practice, the nature of the multivariate functions with multiple local optima prevents this. 
It is much harder to estimate the runtime of the simulated annealing algorithm, as it is entirely reliant on chance in order to even finish running. In practice however, this algorithm is significantly faster than any of the hill climbing methods.
In the worst case, neither algorithm surpasses the complexity class $O\left(\left(\frac{\text{range}}{\varepsilon}\right)^d\right)$, where d is the Number of dimensions, and the range is the difference between the supremum and the infimum. \\
Converting bit strings to the real form is simple. We establish how many bits are necessary to encode the possible values inside of a function's domain, and thus know how many bits are required to build one dimension of the domain, assuming a multi-dimensional domain, as previously mentioned. We generate the integers representing the `index' of the real number, and then build it using the following formula\cite{GA}:

$$ \text{let f be the function to analyse}, f \colon [a, b]^n \to \mathbb{R} $$
$$ \text{let } N = (b - a) \cdot 10^{-\varepsilon}, N = |S_1|, \text{The proof of this fact is trivial} $$
$$ \text{Additionally, but not necessarily important, we have } {N}^n = |S_n| $$
$$ \text{We need } k = \left\lceil \text{log}_2(N)\right\rceil \text{ bits in order to encode one dimension} $$
$$ \text{Bit string length for n dimensions will be } |L| = k \cdot n, \text{where } L = \oplus_{i = 1}^n x_{\text{bits}_i} $$
$$ \text{Assuming } \oplus \text{ to be the concatenation operator}$$
$$ \text{We will define int as the conversion function from bit string to integer} $$
$$ \text{int} \colon \mathbb{B}^n \to \mathbb{N}, \ 
\text{int}(x_\text{bits}) = \sum_{i = 1}^n x_{\text{bits}_i} \cdot 2^{i - 1} $$
$$ \text{Then, } x_{\text{real}_i} = a + \text{int}\left(x_{\text{bits}_i}\right) \cdot \frac{b - a}{2^k - 1} $$

\section{Experiment}
The implementation for the two algorithms was written in C++, compiled using MSVC C++20 in Release mode. All floating-point values used were represented on 64 bits. The RNG method that will be used in all of the algorithms relies on the \underline{\href{https://cplusplus.com/reference/random/mt19937/}{Mersenne Twister}} of the $\langle\text{random}\rangle$ module, that shall be initialised using a \underline{\href{https://en.cppreference.com/w/cpp/numeric/random/seed_seq}{seed sequence}} of 16 bytes generated by a \underline{\href{https://en.cppreference.com/w/cpp/numeric/random/random_device}{random device}}, after which the first 5000 values will be discarded in order to `warm-up` the generator. The timing of all functions was done using a \underline{\href{https://en.cppreference.com/w/cpp/chrono/steady_clock}{steady clock}} of the $\langle\text{chrono}\rangle$ module. The functions were ran at least 40 times in total on each dimension, using each type of improvement, as well as the simulated annealing, on 4 different threads, using \underline{\href{https://en.cppreference.com/w/cpp/thread/async}{asynchronous calls}} from the $\langle\text{future}\rangle$. In order to maintain a faster runtime, when referring to a bit string throughout the paper, we will actually be using an array of booleans, as there is no practical difference between the two methods of bit representation.\\
Each call to the Hill Climbing algorithm executed the improvement function on a randomly generated bit string a number of times $k = \num{e3}$ and returned the best result out of every run. It is important to note that each of the calls were independent of one another. The implementation of the best and worst improvement methods are similar, exploring the entire neighbouring space of each variable. The first improvement version of the implementation has a slightly different implementation, as simply iterating through the bit string until an improvement is found would lead to a bias towards local optima generated by modifying the first half of the bit string. Instead, we must also randomize the order in which each index is being visited each time the neighbouring space is being searched for an improvement. For the sake of showcasing the difference between a randomized selection and a non-randomized one, we implemented both methods.\\
The Simulated Annealing was implemented using both a Geometric-Arithmetic and a Geometric cooling method. The first one works as follows:
$$ \text{let } T \text{ be the current temperature}, T = a \cdot T + b, a = 0.99, b = \num{e-7}, T_0 = \num{e4} $$
The stopping condition used was:
$$ \text{if } T > \num{e-4}\colon\  \text{continue}, \text{ else stop} $$
The second cooling method used was:
$$ \text{let } T \text{ be the current temperature}, T = a \cdot T, a = 0.999, T_0 = \num{e4} $$
The stopping condition for the second method was:
$$ \text{if } T > \num{e-5}\colon\  \text{continue}, \text{ else stop} $$
The probability that a worse neighbouring value will be chosen is: 
$$p = \euler^{-|\frac{x_{\text{new}} - x_{\text{old}}}{T}|}$$
The Geometric Arithmetic implementation will reach a cooled down state in $ k = 1844 $ steps, while the Geometric one will reach a cooled state in $ k = 18412 $ steps. Additionally, for the first cooling method we have chosen to repeat the number of steps $i$ at each new temperature value, where $i$ represents the amount of times the temperature had already been changed. Thus, the algorithm did a total number of steps of $ \sum_{i = 1}^k i $, meaning $\approx \num{17e5} $. For the second cooling method, we have chosen to repeat the number of steps $i \cdot c$ at each new temperature value, where $i$ represents the amount of times the temperature had already been changed and $c$ is a constant $c = 100$. Thus, for the seconf cooling method the algorithm did a total number of steps of $ \sum_{i = 1}^k i \cdot c $, meaning $\approx \num{17e9} $.

\section{Results}
\subsection{5-dimensional results}

\begin{figure}[H]
\begin{tabular}{|c||c|c|c|} \hline
	Function Name & Average f(x) & $\sigma$ of f(x) & Average runtime(s) \\ \hline \hline
	De Jong's & 0 & 0 & 0.146 \\ \hline
	Schwefel's & -2094.7919 & 0.0871 & 0.545 \\ \hline
	Rastrigin's & 0.4975 & 0.5038 & 0.198 \\ \hline
	Michalewicz's & -4.6869 & 0.0010 & 0.668 \\ \hline
\end{tabular}
\caption{Best improvement applied on 5-dimensional functions}
\end{figure}

\begin{figure}[H]
\begin{tabular}{|c||c|c|c|} \hline
	Function Name & Average f(x) & $\sigma$ of f(x) & Average runtime(s) \\ \hline \hline
	De Jong's & 1.0392 & 0.4349 & 0.067 \\ \hline
	Schwefel's & -1642.8964 & 101.9519 & 0.120 \\ \hline
	Rastrigin's & 14.1032 & 3.7678 & 0.070 \\ \hline
	Michalewicz's & -3.4019 & 0.2426 & 0.092 \\ \hline
\end{tabular}
\caption{Randomized first improvement applied on 5-dimensional functions}
\end{figure}

\begin{figure}[H]
    \begin{tabular}{|c||c|c|c|} \hline
        Function Name & Average f(x) & $\sigma$ of f(x) & Average runtime(s) \\ \hline \hline
        De Jong's & 0 & 0 & 0.0936 \\ \hline
        Schwefel's & -2070.35 & 28.9391 & 0.4296 \\ \hline
        Rastrigin's & 0.9923 & 0.5286 & 0.1795 \\ \hline
        Michalewicz's & -3.6987 & 0.0002 & 1.1049 \\ \hline
    \end{tabular}
    \caption{Non-randomized first improvement applied on 5-dimensional functions}
    \end{figure}

\begin{figure}[H]
\begin{tabular}{|c||c|c|c|} \hline
	Function Name & Average f(x) & $\sigma$ of f(x) & Average runtime(s) \\ \hline \hline
	De Jong's & 0 & 0 & 0.302 \\ \hline
	Schwefel's & -1394.9864 & 166.1433 & 0.663 \\ \hline
	Rastrigin's & 1.7925 & 0.6293 & 1.285 \\ \hline
	Michalewicz's & -2.7815 & 0.3264 & 0.017 \\ \hline
\end{tabular}
\caption{Worst improvement applied on 5-dimensional functions}
\end{figure}

\begin{figure}[H]
\begin{tabular}{|c||c|c|c|} \hline
	Function Name & Average f(x) & $\sigma$ of f(x) & Average runtime(s) \\ \hline \hline
	De Jong's & 0 & 0 & 0.826 \\ \hline
	Schwefel's & -2094.3337 & 1.352 & 1.113 \\ \hline
	Rastrigin's & 1.2143 & 0.6362 & 0.891 \\ \hline
	Michalewicz's & -4.6671 & 0.0178 & 1.248 \\ \hline
\end{tabular}
\caption{Simulated Annealing with GA cooling applied on 5-dimensional functions}
\end{figure}

\begin{figure}[H]
    \begin{tabular}{|c||c|c|c|} \hline
        Function Name & Average f(x) & $\sigma$ of f(x) & Average runtime(s) \\ \hline \hline
        De Jong's & 0 & 0 & 0.3077 \\ \hline
        Schwefel's & -2094.693 & 0.4293 & 1.0510 \\ \hline
        Rastrigin's & 0.9885 & 0.6069 & 0.7460 \\ \hline
        Michalewicz's & -3.6925 & 0.0081 & 6.4982 \\ \hline
    \end{tabular}
    \caption{Simulated Annealing with G cooling applied on 5-dimensional functions}
    \end{figure}

\subsection{10-dimensional results}

\begin{figure}[H]
\begin{tabular}{|c||c|c|c|} \hline
	Function Name & Average f(x) & $\sigma$ of f(x) & Average runtime(s) \\ \hline \hline
	De Jong's & 0 & 0 & 1 \\ \hline
	Schwefel's & -4067.7727 & 48.6378 & 4.65 \\ \hline
	Rastrigin's & 3.8668 & 0.7346 & 1.64 \\ \hline
	Michalewicz's & -9.3588 & 0.078 & 5.33 \\ \hline
\end{tabular}
\caption{Best improvement applied on 10-dimensional functions}
\end{figure}

\begin{figure}[H]
\begin{tabular}{|c||c|c|c|} \hline
	Function Name & Average f(x) & $\sigma$ of f(x) & Average runtime(s) \\ \hline \hline
	De Jong's & 11.4351 & 2.9219 & 0.21 \\ \hline
	Schwefel's & -2684.2246 & 634.6331 & 5.14 \\ \hline
	Rastrigin's & 65.3611 & 9.3334 & 0.24 \\ \hline
	Michalewicz's & -4.9379 & 0.3745 & 0.33 \\ \hline
\end{tabular}
\caption{Randomized first improvement applied on 10-dimensional functions}
\end{figure}

\begin{figure}[H]
    \begin{tabular}{|c||c|c|c|} \hline
        Function Name & Average f(x) & $\sigma$ of f(x) & Average runtime(s) \\ \hline \hline
        De Jong's & 0 & 0 & 0.4060 \\ \hline
        Schwefel's & -3911.4676 & 66.1946 & 2.6621 \\ \hline
        Rastrigin's & 5.9521 & 1.0533 & 1.0602 \\ \hline
        Michalewicz's & -8.4160 & 0.0950 & 8.5899 \\ \hline
    \end{tabular}
    \caption{Non-randomized first improvement applied on 10-dimensional functions}
    \end{figure}

\begin{figure}[H]
\begin{tabular}{|c||c|c|c|} \hline
	Function Name & Average f(x) & $\sigma$ of f(x) & Average runtime(s) \\ \hline \hline
	De Jong's & 0 & 0 & 2.01 \\ \hline
	Schwefel's & -1944.4423 & 186.2629 & 4.96 \\ \hline
	Rastrigin's & 7.6206 & 1.3159 & 10.84 \\ \hline
	Michalewicz's & -4.129 & 0.3885 & 0.07 \\ \hline
\end{tabular}
\caption{Worst improvement applied on 10-dimensional functions}
\end{figure}

\begin{figure}[H]
\begin{tabular}{|c||c|c|c|} \hline
	Function Name & Average f(x) & $\sigma$ of f(x) & Average runtime(s) \\ \hline \hline
	De Jong's & 0 & 0 & 1.37 \\ \hline
	Schwefel's & -4169.2265 & 33.2826 & 1.85 \\ \hline
	Rastrigin's & 5.2009 & 1.5669 & 1.48 \\ \hline
	Michalewicz's & -9.3482 & 0.1285 & 2.15 \\ \hline
\end{tabular}
\caption{Simulated Annealing with GA cooling applied on 10-dimensional functions}
\end{figure}

\begin{figure}[H]
    \begin{tabular}{|c||c|c|c|} \hline
        Function Name & Average f(x) & $\sigma$ of f(x) & Average runtime(s) \\ \hline \hline
        De Jong's & 0 & 0 & 0.4919 \\ \hline
        Schwefel's & -4174.6057 & 20.669 & 1.8807 \\ \hline
        Rastrigin's & 5.3156 & 1.7562 & 1.4196 \\ \hline
        Michalewicz's & -8.4015 & 0.1235 & 13.7620 \\ \hline
    \end{tabular}
    \caption{Simulated Annealing with GA cooling applied on 10-dimensional functions}
    \end{figure}

\subsection{30-dimensional results}

\begin{figure}[H]
\begin{tabular}{|c||c|c|c|} \hline
	Function Name & Average f(x) & $\sigma$ of f(x) & Average runtime(s) \\ \hline \hline
	De Jong's & 0 & 0 & 23.74 \\ \hline
	Schwefel's & -11409.7169 & 155.4901 & 135.54 \\ \hline
	Rastrigin's & 28.1095 & 2.6267 & 48.18 \\ \hline
	Michalewicz's & -27.0491 & 0.296 & 128.11 \\ \hline
\end{tabular}
\caption{Best improvement applied on 30-dimensional functions}
\end{figure}

\begin{figure}[H]
\begin{tabular}{|c||c|c|c|} \hline
	Function Name & Average f(x) & $\sigma$ of f(x) & Average runtime(s) \\ \hline \hline
	De Jong's & 114.0378 & 10.1385 & 1.62 \\ \hline
	Schwefel's & -3934.0969 & 341.7946 & 3.41 \\ \hline
	Rastrigin's & 346.8527 & 19.4845 & 1.84 \\ \hline
	Michalewicz's & -8.9402 & 0.5629 & 2.7 \\ \hline
\end{tabular}
\caption{Randomized first improvement applied on 30-dimensional functions}
\end{figure}

\begin{figure}[H]
    \begin{tabular}{|c||c|c|c|} \hline
        Function Name & Average f(x) & $\sigma$ of f(x) & Average runtime(s) \\ \hline \hline
        De Jong's & 0 & 0 & 6.7262 \\ \hline
        Schwefel's & -10754.7054 & 134.9628 & 60.1260 \\ \hline
        Rastrigin's & 36.4184 & 2.4564 & 22.2274 \\ \hline
        Michalewicz's & -25.5630 & 0.3005 & 221.4815 \\ \hline
    \end{tabular}
    \caption{Non-randomized first improvement applied on 30-dimensional functions}
    \end{figure}

\begin{figure}[H]
\begin{tabular}{|c||c|c|c|} \hline
	Function Name & Average f(x) & $\sigma$ of f(x) & Average runtime(s) \\ \hline \hline
	De Jong's & 0 & 0 & 47.83 \\ \hline
	Schwefel's & -3442.8535 & 426.841 & 137.07 \\ \hline
	Rastrigin's & 45.6166 & 3.6626 & 309.83 \\ \hline
	Michalewicz's & -8.1131 & 0.6712 & 0.67 \\ \hline
\end{tabular}
\caption{Worst improvement applied on 30-dimensional functions}
\end{figure}

\begin{figure}[H]
\begin{tabular}{|c||c|c|c|} \hline
	Function Name & Average f(x) & $\sigma$ of f(x) & Average runtime(s) \\ \hline \hline
	De Jong's & 0.0005 & 0 & 3.34 \\ \hline
	Schwefel's & -12242.4767 & 156.8341 & 4.85 \\ \hline
	Rastrigin's & 25.2189 & 7.1704 & 3.66 \\ \hline
	Michalewicz's & -28.1297 & 0.392 & 5.78 \\ \hline
\end{tabular}
\caption{Simulated Annealing with GA cooling applied on 30-dimensional functions}
\end{figure}

\begin{figure}[H]
    \begin{tabular}{|c||c|c|c|} \hline
        Function Name & Average f(x) & $\sigma$ of f(x) & Average runtime(s) \\ \hline \hline
        De Jong's & 0.0001 & 0 & 1.2312 \\ \hline
        Schwefel's & -12332.6682 & 109.7936 & 5.3479 \\ \hline
        Rastrigin's & 23.3521 & 4.9273 & 3.6706 \\ \hline
        Michalewicz's & -27.1150 & 0.3880 & 41.5105 \\ \hline
    \end{tabular}
    \caption{Simulated Annealing with G cooling applied on 30-dimensional functions}
    \end{figure}

\subsection{Interpretation}
On the 5 and 10 dimensional versions of the results, we may notice that the Best Improvement and the Simulated Annealing methods performed similarly, with Simulated Annealing being slightly slower on the 5-dimensional functions and slightly faster on the 10-dimensional functions. When working with the 30 dimensional functions however, their differences become apparent. Simulated Annealing greatly outperforms the hill climbing algorithm, both in runtime and accuracy. Additionally, both first and worst improvement methods were greatly outclassed by the best improvement algorithm, with only an advantage to runtime for the first improvement method, and worst improvement having no redeeming qualities.

\section{Conclusions}
Given the experiment's results, we may conclude the power of both heuristic algorithms when compared to the deterministic exhaustive search alternative that wouldn't have finished running in the following century, with approximative results that rival the real and expected results. This alternative to computing can be expanded as needed, adding or removing precision easily, depending on the time constraints and quality of outputs required. We have noticed the power of the Best Improvement algorithm, but at the cost of a decently high runtime, which we can assume grows with the number of dimensions added to a target function's domain, although it isn't unreasonable in most use-cases. First improvement gave incredibly fast results, but at a huge cost to accuracy. Worst improvement didn't show any advantages, and it may only apply to very specific functions, with further testing required before a conclusion may be drawn. The strongest algorithm, that seemed to improve in both accuracy and speed while increasing the function's domain was Simulated Annealing, which gave better results on the 30-dimensional functions, and approximately equal results to those of Best improvement on the smaller dimensions tested.

\begin{thebibliography}{9}

\bibitem{HC}
``Hill climbing'', Wikipedia, Wikimedia Foundation, 23 March 2023, \url{https://en.wikipedia.org/wiki/Hill\_climbing}

\bibitem{SA}
``Simulated annealing'', Wikipedia, Wikimedia Foundation, 26 August 2023, \url{https://en.wikipedia.org/wiki/Simulated\_annealing}

\bibitem{HD}
``Hamming distance'', Wikipedia, Wikimedia Foundation, 23 September 2023, \url{https://en.wikipedia.org/wiki/Hamming\_distance}

\bibitem{Functions}
Hartmut Pohlheim, ``GEATbx: Example Functions (single and multi-objective functions) 2 Parametric Optimization'', GEATbx, December 2006, \url{http://www.geatbx.com/docu/fcnindex-01.html}

\bibitem{CM}
Mahdi, Walid, Seyyid Ahmed Medjahed, and Mohammed Ouali. ``Performance analysis of simulated annealing cooling schedules in the context of dense image matching." Computación y Sistemas 21.3 (2017): 493-501. \url{https://www.scielo.org.mx/scielo.php?pid=S1405-55462017000300493&script=sci_arttext&tlng=en}

\bibitem{GA}
Eugen Nicolae Croitoru, Department of Computer Science ``Alexandru Ioan Cuza'' University of Iasi, Romania, ``Teaching: Genetic Algorithms'', retrieved October 31, 2023, from \url{https://profs.info.uaic.ro/~eugennc/teaching/ga/}

\end{thebibliography}

\end{document}
